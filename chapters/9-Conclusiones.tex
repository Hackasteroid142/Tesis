\chapter{Conclusiones}
\label{cap:conclusiones}

El método de \textit{self-calibration} implementado es capaz de trabajar tanto con conjunto de datos simulados y reales, de tal manera de afectar el resultado final sin tener que empeorar en gran medida la imagen resultante. Sin embargo, este método implementado no logra la eficiencia que se obtiene de otros métodos de implementación de \textit{self-calibration} como el de CASA, lo que se puede deber que el método implementado carece de algunas funcionalidades que el método de CASA tiene. Esto lleva a que se obtenga un mejor resultado del \textit{self-calibration} de CASA pero aún así el método implementado para este trabajo es un buen hincapié para trabajos futuros, ya que este permite ser mejorado y así obtener una mejor implementación. 

Para el método de \textit{Bispectrum} implementado se logra crear un método que es capaz de almacenar las visibilidades originales y a la vez las visibilidades \textit{bispectrum} junto a los datos esenciales asociados a ella como lo es las antenas, los pesos y \textit{uvw}. Además la implementación de esté método no tarde demasiado en el calculo de las visibilidades \textit{bispectrum} debido a el uso de la antena de referencia, sin embargo, los métodos que utilizan los datos \textit{bispectrum} pueden sufrir de demoras o problemas de memoria debido al tamaño de los datos, conllenvando a errores en los resultados o problemas en la ejecución del código. 

En el caso de la optimización se puede ver que el $\chi^{2}$ es capaz de minimizar pero debido a que los datos son de tamaño muy grande o el calculo del gradiente se tarda demasiado, por lo mismo en futuras implementaciones o mejoras se debería tener en consideración estas falencias de tal manera de optimizar el tiempo de ejecución de este método. Un ejemplo de mejora sería la utilización de la tecnología GPU, ya que en diversas implementaciones u otros estudios se puede observar que se tiene una gran diferencia en rendimiento a diferencia de la implementación en CPU.  

Considerando los objetivos planteados inicialmente para este trabajo se considera que el análisis de la estructura de los datos y pre-procesamiento de estos fue cumplido ya que fue una parte fundamental para la implementación de los métodos de \textit{self-calibration} y \textit{Bispectrum} en el framework \textit{Pyralysis}. Además, en este último método se logra el objetivo de analizar la aplicación a través de optimización o división, debido a que se decanta por este último ya que en el transcurso del desarrollo el método de optimización presentó dificultades que no daban un resultado satisfactorio. 

Ambos métodos de \textit{self-calibration} y \textit{Bispectrum} fueron implementados satisfactoriamente a través de la herramienta Dask, lo que permite un mejor rendimiento y almacenamiento al momento de trabajar con grandes cantidades de datos. Esto permite además que ambos métodos y las funcionalidades necesarias para estos fueran implementadas satisfactoriamente en el framework \textit{Pyralysis} permitiendo su libre uso y mejora. 