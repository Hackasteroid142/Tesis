\chapter{Derivación ecuación \textit{Bispectrum}}
\label{finales:anexo1}

La visibilidad no calibrada puede ser representada mediante la ecuación \Ref{equation:vis_corr}

\begin{equation}
    \tilde{V}_{ij} = g_{i}g_{j}^{*}V_{ij} = |g_{i}g_{j}^{*}V_{ij}|e^{i|\theta_{i} - \theta_{j} + \phi_{ij}|}
    \label{equation:vis_corr}
\end{equation}

donde $\theta_{i} = arg\ g_{i}$ y $\phi_{ij} = arg V_{ij}$. Por lo que $\tilde{\phi_{ij}}$ sería definido según la ecuación \Ref{equation:phi_anex} \citep{Chael_2018}.

\begin{equation}
    \tilde{\phi_{ij}} = arg \tilde{V}_{ij} = \theta_{i} - \theta_{j} + \phi_{ij}
    \label{equation:phi_anex}
\end{equation}

Lo anterior para ser llevado al contexto de \textit{Bispectrum} se debe tener en consideración un conjunto de antenas $i,j,k$, que aplicando la ecuación \Ref{equation:phi_anex} se obtiene lo siguiente para cada par de antenas. 

\begin{equation}
  \label{eq:all_phi}
  \begin{aligned}
    \tilde{\phi_{ij}} = arg \tilde{V}_{ij} = \theta_{i} - \theta_{j} + \phi_{ij}\\        
    \tilde{\phi_{jk}} = arg \tilde{V}_{jk} = \theta_{j} - \theta_{k} + \phi_{jk}\\
    \tilde{\phi_{ki}} = arg \tilde{V}_{ki} = \theta_{k} - \theta_{i} + \phi_{ki}
  \end{aligned}
\end{equation}

Si se suman las relaciones expresadas en \Ref{eq:all_phi} se obtiene lo siguiente.

\begin{equation}
  \label{eq:phi_clear}
  \begin{split}
    \tilde{\phi_{ij}}+\tilde{\phi_{jk}}+\tilde{\phi_{ki}} &=  \theta_{i} - \theta_{j} + \phi_{ij} + \theta_{j} - \theta_{k} + \phi_{jk} + \theta_{k} - \theta_{i} + \phi_{ki} \\
    & = \phi_{ij} + \phi_{jk} + + \phi_{ki} + (\theta_{i} - \theta_{i}) + (\theta_{j} - \theta_{j}) + (\theta_{k} - \theta_{k})\\
    &= \phi_{ij} + \phi_{jk} + + \phi_{ki}
  \end{split}
\end{equation}

En la ecuación \Ref{eq:phi_clear} se pudo eliminar los ángulos que afectaban a la visibilidad observada, dando así a entender que las visibilidades en \textit{Bispectrum} no se ven afectadas por el factor de fase. 
